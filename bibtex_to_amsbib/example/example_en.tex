% !Mode:: "TeX:UTF-8"
%
% The WinEdt Pragma Comments below cause pdflatex + bibtexu commands
% to be executed when PDFTeXify is invoked.
%-------------------------------------------------------------------
% !PDFTeXify engine = pdflatex
% !BIB Exe = bibtexu
% !BIB Opt = -H -l ru -o ru
%-------------------------------------------------------------------
%
% You may also process this file in EVERY system with ARARA by running
% the following command in cmd/terminal: arara ExtremeBarNorms-ru
%
% arara: clean: { extensions: [ aux, bbl, bcf, blg, out, log, run.xml, synctex, synctex.gz, toc ] }
% arara: pdflatex: {draft: yes, shell: yes}
% arara: biber if exists('bcf')
% arara: bibtexu: {options: ['-H', '-l', 'ru', '-o', 'ru']} if found('aux','bibdata') && !exists('bcf')
% arara: pdflatex: {shell: yes, synctex: yes}
% arara: --> until !found('log', '\\(?(R|r)e\\)?run (to get|LaTeX)')
% arara: clean: { extensions: [ aux, bbl, bcf, blg, out, log, run.xml, synctex, synctex.gz, toc ] }
%
\documentclass[a4paper]{article}
\usepackage[T1,T2A]{fontenc}
\usepackage[utf8]{inputenc}
\usepackage[russian,english]{babel}
\usepackage{a4wide}
\usepackage{amsmath,amssymb}
\usepackage[hyper]{amsbib}

\usepackage{fancyvrb}

\title{Bibliography conversion from Bib\TeX{} format to AMSBIB format}

\author{V. Kozyakin\thanks{Institute for Information Transmission Problems, Russian Academy of Sciences, Bolshoj Karetny lane 19, Moscow 127051, Russia,
e-mail: \href{mailto:kozyakin@iitp.ru}{kozyakin@iitp.ru}}}
%\date{}

\begin{document}
\maketitle

When preparing manuscripts for publication in the vast majority (more than 150) of Russian mathematical journals, the \href{https://www.mathnet.ru/index.phtml?&option_lang=eng}{Math-Net.Ru} portal recommends that the bibliography be represented in the \href{https://www.mathnet.ru/poffice/amsbibpackage.phtml?wshow=amsbibpackage&option_lang=eng}{AMSBIB} format.

If references to publications in Russian-language journals are indexed in Math-Net.Ru, there is no particular problem since the corresponding bibliographic records in the AMSBIB format can be copied from the corresponding pages of publications on the site \href{https://www.mathnet.ru/index.phtml?&option_lang=eng}{Math-Net.Ru}. The situation is worse with references to English-language publications, most of which are not indexed on Math-Net.Ru site, and for which, accordingly, bibliographic information in the AMSBIB format, as a rule, is not available. In this case, one has to manually compile the corresponding bibliographic records in the AMSBIB format, using widely available ones (for example, on the site \href{https://mathscinet.ams.org/mrlookup}{MR Lookup}, on journal sites, or on numerous bibliographic Internet services) corresponding bibliographic records in Bib\TeX format.

Unfortunately, a direct correspondence between the fields of bibliographic records in the AMSBIB and Bib\TeX{} formats does not exist. As a result, the process of translating records from one format to another becomes a complex and creative endeavor. When translating a limited number of publications, there are no significant challenges. However, when faced with the task of translating a substantial amount of bibliographic records from Bib\TeX{} to AMSBIB (such as in the preparation of a review or monograph), the process becomes arduous. Not only is manual translation prone to numerous errors, but it also heavily relies on the individual author's creativity.

To simplify and unify the process of converting bibliography from the Bib\TeX{} format to the AMSBIB format, I created \texttt{amsbib.bst} and \texttt{amsbibs.bst} style files that perform this conversion automatically. Moreover, the first of these style files creates a list of AMSBIB bibliographic records in the order of citation of publications in the work, and the second one in alphabetical order.

An example of such a transformation is given in the listing below, and its result is at the end of this work:
\begin{Verbatim}[frame=lines,fontsize=\small,numbers=left,label=Fragment of the example tex-file,framesep=12pt]
\documentclass[a4paper]{article}
\usepackage[T1,T2A]{fontenc}
\usepackage[utf8]{inputenc}
\usepackage[english,russian]{babel}
\usepackage{amsmath,amssymb}
\usepackage[hyper]{amsbib}
......................................
......additional preamble stuff.......
......................................
\title{....}
\author{....}

\begin{document}
\maketitle
......................................
...........publication text...........
......................................
\nocite{*}

\bibliographystyle{amsbib}
\bibliography{example}
\end{document} 
\end{Verbatim}

In this case, the bibliography itself (created using the \texttt{amsbib.sty} package) is both inserted into the pdf file created during the translation of the tex file, and placed into the \texttt{<file name>.bbl} file generated during the translation tex file.

We emphasize that both the bibliography file \texttt{.bib} and the tex-file using it must be in the same encoding. For example, in this work, \texttt{utf8} encoding was used. In the case of \texttt{cp866} or \texttt{cp1251} encodings, the \texttt{bibtex8} program should be used to process the bibliography, and when \texttt{utf8} encoding is used, the \texttt{bibtexu} program should be used.

The proposed style files \texttt{amsbib.bst} and \texttt{amsbibs.bst} are far from being perfect, they are only the first attempt in this direction. Therefore, \textbf{it is recommended that the resulting list of bibliographic records in AMSBIB format be carefully checked and, if necessary, corrected manually.}

\sloppy Style files \texttt{amsbib.bst} and \texttt{amsbibs.bst}, and example files \texttt{example.tex} and \texttt{\detokenize{example_en.tex}} can be downloaded from \href{https://github.com/kozyakin/kozyakin.github.io/tree/main/bibtex_to_amsbib}{BibTeX to AMSBIB} of my GitHub~Pages repository \href{https://kozyakin.github.io/}{kozyakin.github.io}. The files of the AMSBIB package (\texttt{amsbib.sty} + \texttt{*.pdf}) required for translating examples are borrowed from 
\href{https://www.mathnet.ru/poffice/amsbib.zip}{amsbib.zip}.
\bigskip

The following is an excerpt from the bibliography database \texttt{amsbib.bib} in the Bib\TeX{} format used in this example:

\begin{Verbatim}[frame=lines,fontsize=\small,label=Fragment of the BibTeX database amsbib.bib,framesep=12pt]
@ARTICLE{BKK:IEEETNN96,
 author        = "Bhaya, Amit and Kaszkurewicz, Eugenius and Kozyakin, V. S.",
 title         = "Existence and stability of a unique equilibrium in
                  continuous-valued discrete-time asynchronous {H}opfield
                  neural networks",
 journal       = "IEEE Trans. Neural Netw.",
 fjournal      = "IEEE Transactions on Neural Networks",
 year          = "1996",
 volume        = "7",
 number        = "3",
 pages         = "620--628",
 month         = may,
 issn          = "1045-9227",
 doi           = "10.1109/72.501720",
 url           = "https://ieeexplore.ieee.org/document/501720",
 language      = "english",
}

@ARTICLE{ChadKra:APM2:97,
 author        = "Ch{\k{a}}dzy{\'n}ski, Jacek and Krasi{\'n}ski, Tadeusz",
 title         = "A set on which the {{\L}}ojasiewicz exponent at infinity is
                  attained",
 journal       = "Ann. Polon. Math.",
 fjournal      = "Annales Polonici Mathematici",
 year          = "1997",
 volume        = "67",
 number        = "2",
 pages         = "191--197",
 eprinttype    = "arXiv",
 eprint        = "math/9802064",
 coden         = "APNMA4",
 issn          = "0066-2216",
 mrclass       = "14E05",
 mrnumber      = "1460600 (98j:14013)",
 mrreviewer    = "Zbigniew Jelonek",
 language      = "english",
}

.......................................................

@BOOK{AizGant:r,
 author        = "Айзерман, М. А. and Гантмахер, Ф. Р.",
 title         = "Абсолютная устойчивость регулируемых систем",
 publisher     = "Изд-во АН СССР",
 address       = "М.",
 year          = "1963",
 pagetotal     = "140",
 language      = "russian",
}

@ARTICLE{Anosov:PSIM67:r,
 author        = "Аносов, Д. В.",
 title         = "Геодезические потоки на замкнутых римановых многообразиях
                  отрицательной кривизны",
 journal       = "Тр. МИАН",
 fjournal      = "Труды Математического института имени В. А. Стеклова",
 year          = "1967",
 volume        = "90",
 pages         = "3--209",
 url           = "https://mi.mathnet.ru/tm2795",
 language      = "russian",
}

........................................................
\end{Verbatim}
\bigskip

The following is a fragment of the \texttt{example.bbl} file generated as a result of the conversion and containing the bibliography database in the AMSBIB format:

\begin{Verbatim}[frame=lines,fontsize=\small,label=Fragment of the AMSBIB database \detokenize{example_en.bbl} generated during conversion,framesep=12pt]
\begin{thebibliography}{10}
% \bib, bibdiv, biblist are defined by the amsrefs package.

\Bibitem{BKK:IEEETNN96}
\by A.~Bhaya, E.~Kaszkurewicz, V.~S.~Kozyakin
\paper  Existence and stability of a unique equilibrium in continuous-valued
  discrete-time asynchronous {H}opfield neural networks
\jour IEEE Trans. Neural Netw.
\yr 1996
\vol 7
\issue 3
\monthissue May
\pages 620--628
\crossref{https://dx.doi.org/10.1109/72.501720}
\elink{\url{ https://ieeexplore.ieee.org/document/501720}}

\Bibitem{ChadKra:APM2:97}
\by J.~Ch{\k{a}}dzy{\'n}ski, T.~Krasi{\'n}ski
\paper  A set on which the {{\L}}ojasiewicz exponent at infinity is attained
\jour Ann. Polon. Math.
\yr 1997
\vol 67
\issue 2
\pages 191--197
\arxiv \href{http://arXiv.org/abs/math/9802064}{\allowbreak
  math/9802064}\miscnote
\mathscinet{https://www.ams.org/mathscinet-getitem?mr=1460600}

.......................................................

\RBibitem{AizGant:r}
\by М.~А.~Айзерман, Ф.~Р.~Гантмахер
\book Абсолютная устойчивость регулируемых
  систем
\yr 1963
\publ Изд-во АН СССР
\publaddr М.
\totalpages 140

\RBibitem{Anosov:PSIM67:r}
\by Д.~В.~Аносов
\paper  Геодезические потоки на замкнутых
  римановых многообразиях отрицательной
  кривизны
\jour Тр. МИАН
\yr 1967
\vol 90
\pages 3--209
\mathnet{https://mi.mathnet.ru/tm2795}

........................................................

\end{thebibliography}
\end{Verbatim}
\fussy\nocite{*}

\bibliographystyle{amsbibs}
\bibliography{example}
\end{document} 